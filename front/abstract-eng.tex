\pagestyle{empty}
\begin{latin}
\begin{center}
\textbf{Abstract}
\end{center}
\baselineskip=.8\baselineskip

Smart devices, such as sensors, generate a wide range of data in smart homes. This collected data enables the provision of value-added services and desired features to smart home owners through IoT platforms and cloud computing systems. Numerous solutions have been proposed to protect user privacy in the IoT, including methods to counteract eavesdropping on the traffic sent from smart homes by attackers or to limit IoT platforms' access to users' sensitive information. However, particular attention has been given to solutions that focus on preserving user privacy without restricting the services users receive from cloud computing platforms. In this research, we have designed and implemented a solution based on smart home ontology that preserves user privacy while not affecting the services received from the platforms. In the proposed solution, by injecting and sending fake traffic to the IoT platform, the ability of an honest-but-curious platform to distinguish the user's real activities is negated. These fake activities are generated based on the home ontology and the user's activity history, making them indistinguishable from real activities from the platform's perspective. In this approach, the user can achieve their desired level of privacy by identifying and replacing sensitive activities. To evaluate and verify the effectiveness of the proposed solution, classification techniques have been used to detect activities and measure their accuracy.

\bigskip\noindent\textbf{Keywords}: Internet of Things, Privacy, Smart Home, Ontology, IoT Platform

\end{latin}
