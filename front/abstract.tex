\شروع{وسط‌چین}
\مهم{چکیده}
\پایان{وسط‌چین}
\بدون‌تورفتگی

تجهیزات هوشمند، مانند حسگرها، در خانه‌های هوشمند مجموعه گسترده‌ای از داده‌ها را تولید می‌کنند. این داده‌های جمع‌آوری‌شده امکان ارائه خدمات ارزش افزوده و سرویس‌های دلخواه را از طریق سکوهای اینترنت اشیاء و سیستم‌های رایانش ابری برای صاحبان خانه‌های هوشمند فراهم می‌سازد. تاکنون راهکارهای متعددی برای حفظ حریم خصوصی کاربران در اینترنت اشیاء پیشنهاد شده است؛ از جمله راهکارهایی برای مقابله با شنود ترافیک ارسالی خانه‌های هوشمند توسط مهاجمان و یا محدودسازی دسترسی سکوهای اینترنت اشیاء به اطلاعات حساس کاربران. با این حال، توجه خاصی به راهکارهایی شده است که بدون محدودسازی خدماتی که کاربر از سکوهای رایانش ابری دریافت می‌کند، بر حفظ حریم خصوصی کاربر متمرکز هستند. در این پژوهش، ما راه حلی مبتنی بر هستی‌شناسی خانه هوشمند طراحی و پیاده‌سازی کرده‌ایم که ضمن حفظ حریم خصوصی کاربر، تأثیری بر خدمات دریافتی او از سکوها نمی‌گذارد. در راهکار پیشنهادی این پژوهش، با تزریق و ارسال ترافیک جعلی به سکوی اینترنت اشیاء، امکان تشخیص فعالیت واقعی کاربر از سکوی صادق اما کنجکاو سلب می‌شود. این فعالیت‌های جعلی بر اساس هستی‌شناسی خانه و سوابق فعالیت کاربر ایجاد می‌شوند تا از دید سکو از فعالیت واقعی قابل تمایز نباشند. در این راهکار، کاربر با تعیین و جایگزینی فعالیت‌های حساس، به سطح حریم خصوصی دلخواه خود دست می‌یابد. برای ارزیابی و تأیید صحت عملکرد راهکار پیشنهادی، از دسته‌بندی برای تشخیص فعالیت‌ها و اندازه‌گیری دقت آن استفاده شده است.

\پرش‌بلند
\بدون‌تورفتگی \مهم{کلیدواژه‌ها}: اینترنت اشیاء، حریم خصوصی، خانه‌ی هوشمند، هستی‌شناسی، سکوهای اینترنت اشیاء 

\صفحه‌جدید