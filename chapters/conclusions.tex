\فصل{نتیجه‌گیری}

در این پژوهش راهکاری مبتنی بر تولید رویداد جعلی برای حفظ حریم خصوصی کاربر در برابر سکوی نامعتمد در خانه هوشمند ارائه شد. این راهکار برای فریب سکو و پنهان‌سازی فعالیت‌های حساس کاربر، از تولید سلسله فعالیت جعلی بنا به درخواست کاربر بهره می‌برد تا سکوی اینترنت اشیاء متوجه تفاوت بین فعالیت‌های جعلی و واقعی نشود. در ادامه ضمن جمع‌بندی، به طرح پیشنهادهایی برای پژوهش‌های آینده و تکمیل راهکار خواهیم پرداخت.

\section{جمع‌بندی}

هدف این پژوهش تولید سلسله فعالیت جعلی برای حفظ حریم خصوصی کاربر بوده که برای این هدف از روش تولید رویداد جعلی بهره برداری شده است. تولید رویداد جعلی بر اساس هستی‌شناسی خانه هوشمند بوده که دانش اولیه‌ی آن توسط فرد خبره با روش دانش‌محور فراهم شده است. در این پژوهش از هستی‌شناسی برای مدل‌سازی معنایی رویدادهای مربوط به دستگاه‌های اینترنت اشیاء و سناریوهای رفتارهای کاربران در محیط خانه هوشمند استفاده شد تا بر مبنای آن امکان تولید رویدادها و رفتارهای جعلی به صورت غیرقابل تمایز با رویدادها و رفتارهای واقعی در محیط خانه هوشمند فراهم گردد.

برای عدم تشخیص سکوی نامعتمد، از عوامل تصادفی‌ساز در انجام و زمان‌بندی هر رویداد جعلی استفاده شده و هر رویداد توسط نرم‌افزار، برچسب جعلی می‌خورد و به سکو ارسال می‌شود. کنش مربوط به رویدادها توسط نرم‌افزار نظارت شده و کنش‌های مربوط به رویدادهای جعلی کنار گذاشته می‌شود تا بهره‌وری خانه هوشمند پایین نیاید.

رویکرد استفاده شده در این پژوهش که بر مبنای هستی‌شناسی بوده، تا کنون کمتر مورد توجه قرار گرفته است و امید است تا حریم خصوصی کاربران در خانه هوشمند با استفاده از نتایج این پژوهش، حفظ شود. بر اساس نتایج به دست آمده، دسته‌بند مورد ارزیابی که دقت تشخیص فعالیتش برای مجموعه داده \lr{Orange4Home} برابر با 98 درصد بود، در صورت دریافت سلسله فعالیت جعلی که با استفاده از راهکار پیشنهادی این پژوهش تولید شده است دقتش به طور میانگین برابر با ....... درصد است که عدم کاهش دقت در تشخیص فعالیت به معنی نزدیک به واقعیت بودن سلسله فعالیت جعلی است و در صورت دریافت سلسله فعالیت جعلی و واقعی به همراه یکدیگر، دقتش برابر با ....... درصد است که کاهش دقت در تشخیص فعالیت به معنی تمایزناپذیری فعالیت‌های جعلی از فعالیت‌های واقعی می‌باشد.

\section{پیشنهادهایی برای پژوهش‌های آینده}

جهت تکمیل راهکار ارائه شده در این پژوهش جهت حفظ حریم خصوصی کاربر در خانه هوشمند، پیشنهادهای زیر ارائه می‌گردد:

\begin{itemize}
\item \textbf{ذخیره‌ی وضعیت موجودیت‌ها از دیدگاه سکو}: اگر سکوی اینترنت اشیاء توانایی ذخیره و استنتاج وضعیت فعلی موجودیت‌های خانه هوشمند را داشته باشد، پس از مدتی به دلیل متغایر بودن وضعیت فعلی یک موجودیت و وضعیت مورد انتظار، توانایی تشخیص جعلی بودن آخرین فعالیتی که اقدام به تغییر وضعیت آن موجودیت کرده را دارد. برای حل این مشکل، نیاز است تا راهکارهای تولیدکننده سلسله فعالیت جعلی علاوه بر استفاده از هستی‌شناسی، وضعیت مورد انتظار هر موجودیت از دید سکو را ذخیره کرده و سلسله فعالیت جعلی را بر اساس آن تولید کند.
\item \textbf{ترکیب روش دانش‌محور ارائه شده با روش‌های داده‌محور}: رفتار کلی افراد در خانه هوشمند با گذر زمان تغییر می‌کند و سکو همواره در حال به‌روزرسانی اطلاعات خود از خانه هوشمند است. هستی‌شناسی فعالیت افراد در خانه هوشمند و احتمال توالی آن‌ها که در قوانین انجمنی و تولید سلسله فعالیت جعلی مورد استفاده قرار می‌گیرد می‌تواند با استفاده از روش‌های داده‌محور به‌روزرسانی شده تا سلسله فعالیت جعلی مبتنی بر آخرین فعالیت‌های افراد باشند. برای نزدیک بودن هر چه بیشتر سلسله فعالیت‌ جعلی تولید شده می‌توان به فعالیت‌های جدیدتر وزن بیشتری اختصاص داد تا مبنای اصلی آخرین فعالیت‌ها باشند. به‌روزرسانی هستی‌شناسی به صورت روزانه و حتی هفتگی توسط فرد خبره امری دشوار است که می‌توان با استفاده از روش‌های دانش‌محور این کار را به صورت خودکار انجام داد.
‌\end{itemize}
