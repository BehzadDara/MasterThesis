\فصل{نتیجه‌گیری}

در این پژوهش راهکاری مبتنی بر تولید رویداد جعلی برای حفظ حریم خصوصی کاربر در برابر سکوی نامعتمد در خانه هوشمند ارائه شد. این راهکار برای فریب سکو و پنهان‌سازی فعالیت‌های حساس کاربر، از تولید سلسله فعالیت جعلی بنا به درخواست کاربر بهره می‌برد تا سکوی اینترنت اشیاء متوجه تفاوت بین فعالیت‌های جعلی و واقعی نشود. در ادامه ضمن جمع‌بندی، به طرح پیشنهادهایی برای پژوهش‌های آینده و تکمیل راهکار خواهیم پرداخت.

\section{جمع‌بندی}

هدف این پژوهش تولید سلسله فعالیت جعلی برای حفظ حریم خصوصی کاربر بوده که برای این هدف از روش تولید رویداد جعلی بهره برداری شده است. تولید رویداد جعلی بر اساس هستی‌شناسی خانه هوشمند بوده که دانش اولیه‌ی آن توسط فرد خبره با روش دانش‌محور فراهم شده است.

برای عدم تشخیص سکوی نامعتمد، از عوامل تصادفی‌ساز در انجام و زمان‌بندی هر رویداد جعلی استفاده شده و هر رویداد توسط نرم‌افزار، برچسب جعلی می‌خورد و به سکو ارسال می‌شود. کنش مربوط به رویدادها توسط نرم‌افزار نظارت شده و کنش‌های مربوط به رویدادهای جعلی کنار گذاشته می‌شود تا بهره‌وری خانه هوشمند پایین نیاید.

رویکرد استفاده شده در این پژوهش که بر مبنای هستی‌شناسی بوده، تا کنون کمتر مورد توجه قرار گرفته است و امید است تا حریم خصوصی کاربران در خانه هوشمند با استفاده از نتایج این پژوهش، حفظ شود.

\section{پیشنهادهایی برای پژوهش‌های آینده}

جهت تکمیل راهکار ارائه شده در این پژوهش جهت حفظ حریم خصوصی کاربر در خانه هوشمند، پیشنهادهای زیر ارائه می‌گردد:

\begin{itemize}
\item \textbf{ذخیره‌ی وضعیت موجودیت‌ها از دیدگاه سکو}: چنانچه سکوی اینترنت اشیاء، وضعیت و موقعیت موجودیت‌ها را ذخیره و روی آن‌ها استنتاج کند؛ می‌تواند متوجه جعلی بودن سلسله فعالیت شود. با استفاده از نرم‌افزار این پژوهش، بعد از مدت زمانی سکو متوجه جعلی بودن سلسله فعالیت‌های قبلی می‌شود اما امکان تشخیص این که کدام سلسله فعالیت جعلی بوده را ندارد. حتی در صورتی که بتواند با استنتاج روی کنش‌های اعمال نشده متوجه این امر شود، کماکان برای سلسله فعالیت‌های آینده امکان تشخیص در آن لحظه را ندارد زیرا هر بار سلسله فعالیتی متنوع با استفاده از عوامل تصادفی‌ساز تولید می‌شود. حال اگر نیاز به حل این مشکل باشد، نرم‌افزار باید وضعیت کامل خانه هوشمند از دید سکو را به طول کامل و با جزئیات ذخیره کند تا تولید سلسله فعالیت‌های جعلی بعدی و ترکیب آن با سلسله فعالیت‌های واقعی را بتواند انجام دهد تا تشخیص جعلی بودن یک سلسله فعالیت از دید سکو دشوارتر و حتی غیرممکن شود.
%\item \textbf{خودکارسازی تولید سلسله فعالیت جعلی}: در این پژوهش تولید سلسله فعالیت جعلی تنها به درخواست کاربر صورت می‌گیرد. می‌توان برای خودکارسازی این امر، با توجه به فعالیت‌های حساس کاربر و در جهت حفظ حریم خصوصی، نرم‌افزار به صورت هوشمند سلسله فعالیت جعلی تولید و به سکو ارسال کند تا نیازی به ورودی‌های کاربر به صورت مداوم وجود نداشته باشد.
\item \textbf{استفاده از روش ترکیبی در ارائه دانش}: برای تکمیل هستی‌شناسی خانه هوشمند، می‌توان تشخیص فعالیت کاربران را پس از ارائه دانش توسط فرد خبره، همواره کامل‌تر کرد که تکمیل تشخیص فعالیت کاربران با استفاده از روش داده‌محور است. این تکمیل هستی‌شناسی، برای تنوع در سلسله فعالیت‌های جعلی و به روز بودن نرم‌افزار از رفتار کاربران است که در تکمیل این پژوهش می‌تواند عمل کند.
‌\end{itemize}
