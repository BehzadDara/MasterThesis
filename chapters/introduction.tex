\فصل{مقدمه}

\پاورق{خانه‌های هوشمند}{Smart homes} نمونه‌ای از کاربردهای مبتنی بر فناوری‌های نوین، برای کمک به زندگی مستقل جمعیت مسن رو به رشد در جهان و همچنین بالاتر بردن کیفیت زندگی انسانها از بعد راحتی و آسایش و \پاورق{‌امنیت}{Safety} هستند. با اینکه از تعریف اولیه این مفهوم بیش از 20 سال می‌گذرد اما با پیشرفت فناوری \پاورق{‌حسگرها}{Sensors} و کوچکتر شدن اندازه آن‌ها و همچنین ارزانتر شدن هزینه‌های استفاده از آن‌ها در خانه‌های هوشمند، تحقیقات و پیشرفت‌های این حوزه در حال سرعت گرفتن است. خانه‌های هوشمند در کنار کاربردهای دیگر \پاورق{‌اینترنت اشياء}{Internet of things} مانند کشاورزی هوشمند، سیستم‌های مبتنی بر سلامت هوشمند و امثالهم، باعث افزایش تعداد حسگرها و عملگرهای اینترنت اشیاء به کارگرفته شده در جهان شده‌اند. طبق برآورد صاحب نظران اين حوزه ﺗﺎ ﺳﺎل 2025 ﺗﻌﺪاد دﺳﺘﮕﺎهﻫﺎی اﯾﻨﺘﺮﻧﺖ اﺷﯿﺎء در ﺣﺎل اﺳﺘﻔﺎده در ﺟﻬﺎن، ﺑﻪ ﺑﯿﺶ از ۵۷ ﻣﯿﻠﯿﺎرد ﺧﻮاﻫﺪ رسید \cite{x11}.

عمده تحقیقات و پیشرفت‌های صورت گرفته در حوزه خانه‌های هوشمند، جدا از بهبودها و پیشرفت‌های صورت گرفته در حوزه حسگرها، در زیرحوزه‌های سلامت انسان‌ها مانند پایش اطلاعات حیاتی مرتبط با بیماران، پایش رفتار افراد مسن، \پاورق{‌خودکارسازی}{Automation} رفتارها، کنترل انرژی‌های مصرفی در خانه و دسترسی به سرویس‌های از راه دور صورت گرفته است که همه آن‌ها متکی بر شناسایی و کلاس‌بندی رفتار کاربران است.

با پیشرفت‌های هرچه بیشتر در این حوزه، به تدريج مشکلات بیشتری نیز مربوط به چگونگی حفظ حریم‌ خصوصی کاربر شناسایی و معرفی می‌گردد. برای مثال با بیشتر شدن استفاده از تجهیزات هوشمند بی‌سیم در خانه‌های هوشمند، آسیب‌پذیری‌های ذاتی این تجهيزات و حملات مرتبط با آن‌ها از جمله \پاورق{‌حمله کانال جانبی}{Side channel attack} که مربوط به شنود ترافیک ارسالی و استنتاج اطلاعات حساس به صورت غیر مستقیم از ترافیک ارسالی است، حریم خصوصی کاربر را بیشتر در معرض خطر قرار داده است \cite{x12}. 

از سوی ديگر، تجهیزات اینترنت اشیاء موجود در بازار که قابل استفاده در خانه‌های هوشمند هستند همگی ساخت یک تولید کننده خاص نیستند و لذا با مشکل عدم امکان ﺗﻌﺎﻣﻞ با یکديگر روبرو هستند. این مشکل، کاربران را متمایل به استفاده از ﺳﮑﻮﻫﺎی اﯾﻨﺘﺮﻧﺖ اﺷﯿﺎء می‌نمايد چرا که این \پاورق{‌سکوهای اینترنت اشياء}{Internet of things platforms}، ﮐﺎرﺑﺮان را ﻗﺎدر ﻣﯽﺳﺎزﻧﺪ ﺗﺎ ﺑﺎ اﺗﺼﺎل دﺳﺘﮕﺎهﻫﺎ و ﺳﺮوﯾﺲﻫﺎی ﺑﺮﺧط ﮔﻮﻧﺎﮔﻮن ﺑﻪ ﯾﮑﺪﯾﮕﺮ، قواعد ﺧﻮدﮐﺎرﺳﺎزی دﻟﺨﻮاه خود را اﻋﻤﺎل ﮐﻨﻨﺪ و از سرویس‌های متنوع ارائه شده توسط این سکوها بهره‌مند شوند برای مثال یکی از سرویس‌های مورد استقبال کاربران در این حوزه، شناسایی رفتار فعلی کاربر و ارائه پاسخ دقیق به کاربر در مقابل رفتار مشاهده شده است.

از آنجا که ﺳﮑﻮﻫﺎی اینترنت اشیاء ﻫﯿﭻ ﻗﺎﺑﻠﯿﺘﯽ ﺑﺮای ﮐﻨﺘﺮل ﻧﺸﺖ دادهﻫﺎی حسگرها، در اﺧﺘﯿﺎر ﮐﺎرﺑﺮان ﻗﺮار ﻧﻤﯽدﻫند، لذا حریم خصوصی کاربر را با خطر مواجه می‌نمایند. هنگامی که از امکان نقض حریم خصوصی کاربر با دسترسی غیر مجاز به داده‌های رفتاری کاربر حاصل از حسگرهای خانه‌های هوشمند صحبت می‌کنیم در واقع به این موضوع توجه داریم که تجهیزات یک خانه هوشمند همچون حسگرها و عملگرها، طیف وسیعی از داده‌های رفتاری ساکنان خانه هوشمند را به طور منظم جمع‌آوری می‌نمایند. به عبارت دیگر تجهیزات هوشمند امروزی مانند گوشی‌های تلفن همراه، ساعت‌های هوشمند، \پاورق{‌تجهیزات پوشیدنی}{Bearable device} هوشمند و بسیاری از تجهیزات الکترونیکی مدرن، قابلیت تولید داده دارند. چون این داده‌ها در قالب‌های خام و اولیه خود شامل اطلاعات حساسی درباره ساکنان خانه هوشمند هستند و همچنین با توجه به اينکه در زمانی زندگی می‌کنیم که جرایم سایبری هر روز گسترده‌تر، ويرانگرتر و پيچيده‌تر می‌شود، لذا جمع‌آوری داده‌ها بدون توجه کافی به نوع و مفهوم داده‌های ارسالی از دیدگاه مهاجمین، تبعات حتمی نقض حریم خصوصی کاربر و استفاده غیر مجاز از این داده‌ها را به دنبال خواهد داشت. به همین دلیل است که طبق مطالعات صورت گرفته اخیر، حفظ حریم خصوصی کاربران یکی از موانع بسیار اساسی در توجه و سازگار شدن عموم افراد به استفاده از فناوری‌های خانه‌های هوشمند است \cite{x13}. 

با توجه به مواردی که ذکر شد مشخص است که تحلیل قابل اعتماد داده‌های ارسالی حسگرها و عملگرها و به طور کلی رفتار کاربر در یک خانه هوشمند و کسب اطمینان از محافظت از این داده‌ها در مقابل دسترسی مهاجمینی که اقدام به شنود ترافیک ارسالی می‌نمایند و یا عدم ارسال داده‌های محرمانه کاربر به سکوهای اینترنت اشیاء، چالش بزرگی پیش روی ارائه کنندگان راهکارهای امنیتی در این حوزه است.

در حوزه حفظ حریم خصوصی کاربر در برابر سکوهای نامعتمد اینترنت اشیاء، این سوال مطرح است که چگونه می‌توان داده‌های حسگرها را به سکوهای اینترنت اشیاء ارسال کرد و از سرویس‌های متنوع این سکوها بهره‌مند شد بدون این که به حریم خصوصی کاربر خدشه‌ای وارد شود و فعالیت‌های حساس و رفتار کاربر از دید سکو قابل شناسایی نباشد. راهکارهای ارائه شده برای پاسخ به اين سوال می‌بایست توازنی در پاسخ به هر دو مسئله داشته باشند و مصالحه‌ای بین حفظ حریم خصوصی کاربر و دریافت سرویس‌های مد نظر کاربر در خانه هوشمند ایجاد نمایند. این راهکارها می‌بایست برای شناسایی رفتار کاربر در خانه هوشمند یک مدل رفتاری مناسب ایجاد نمایند و سپس قادر باشند تا با پنهان‌سازی، رفتارهای حساس کاربر را از دیدگاه سکوهای اینترنت اشیاء، مخفی نمایند. 

در سال‌های اخیر، راهکارهایی در جهت حفظ حریم خصوصی کاربر در خانه‌های هوشمند در برابر سکوهای نامعتمد ارائه شده است. این راهکارها بر اساس روش، به راهکارهای مبتنی رمزنگاری، \پاورق{‌کمینه‌سازی}{Filtering}، \پاورق{‌آشفته‌سازی}{Randomization} و تولید رویداد جعلی تقسیم می‌شوند. راهکارهای مبتنی بر رمزنگاری، با استفاده از تکنیک‌های رمزنگاری، محاسبات چندجانبه امن و محیط اجرای امن داده‌های کاربر را از دید سکو پنهان می‌کنند. راهکارهای مبتنی بر کمینه‌سازی، از روش حذف داده‌هایی که در اجرای فواعد \پاورق{‌رهانا}{Trigger}-\پاورق{‌کنش}{Action} اثرگذار نیستند، اقدام به کاهش اطلاعات ارسالی به سکو و ناقص کردن دانش آن می‌کنند. راهکارهای مبتنی بر آشفته‌سازی، داده‌های کاربر را قبل از ارسال به سکو به شکل‌های مختلف تغییر می‌دهند. راهکارهای مبتنی بر رویداد جعلی، برای حفظ حریم خصوصی کاربر و اطلاعات حساس آن، از ارسال رویدادهای جعلی به سکو استفاده می‌کند؛ به نحوی که از دید سکو رویدادهای جعلی و واقعی قابل تمایز نباشند.

این پژوهش با هدف افزایش امنیت در خانه‌های هوشمند انجام شده و از \پاورق{‌هستی‌شناسی}{Ontology} خانه‌های هوشمند بهره برده است. در این راستا، برای محافظت از امنیت خانه‌های هوشمند در برابر حملات مخرب و جلوگیری از نفوذ مهاجمان، اقدام به تولید سلسله رویداد جعلی شده است. این سلسله رویدادها با دقت و اصول هستی‌شناسی خانه طراحی شده‌اند به نحوی که مهاجمان قادر به تشخیص دقیق داده‌های واقعی از داده‌های جعلی نباشند و به تبع آن، نتوانند اطلاعات حساس مربوط به زندگی افراد در خانه‌های هوشمند را به دست آورند.

یکی از جوانب مهم در طراحی این راه حل این است که تنوع و تصادف در تولید سلسله رفتارها حفظ شده و از الگوهای قابل پیش‌بینی پرهیز گردد. برای این منظور، از \پاورق{‌عوامل تصادفی‌ساز}{Randomizing factors} بهره گرفته شده تا مهاجمین نتوانند با تحلیل تکراری بودن رفتارها به اهداف خود دست یابند. اقدام دیگری که علاوه بر تولید متنوع سلسله رفتارها برای گمراه‌سازی مهاجم انجام می‌شود، زمان انجام هر رفتار پس از رفتار دیگر است که با استفاده از عوامل تصادفی ساز، زمان انجام هر رفتار در بازه‌ای مشخص متغیر است. این پژوهش امیدوار است که با اجرای این برنامه، امنیت خانه‌های هوشمند تقویت شده و از حملات ناخواسته جلوگیری شود.

این پایان‌نامه در شش فصل ابعاد مختلف مساله را بررسی کرده و ارائه‌ی راه‌حل و ارزیابی آن را انجام می‌دهد. در فصل دوم تعاریف مفاهیم پایه‌ی مورد نیاز برای درک کامل مساله ارائه می‌شود، در فصل سوم پژوهش‌های پیشین مرتبط با این پژوهش را بررسی کرده که هر یک به بررسی یک یا چند بخش مرتبط با این پژوهش را انجام داده‌اند. در فصل چهارم راه‌حل ارائه شده برای حل این مساله را مدل‌سازی کرده و پیاده سازی کامل و جامع آن را ارائه می‌کنیم. در فصل پنجم به ارزیابی روش پیشنهادی و ارائه نتایج حاصل از ارزیابی می‌پردازیم و در فصل آخر نتیجه‌گیری این پژوهش ارائه خواهد شد.

